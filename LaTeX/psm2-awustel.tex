\documentclass[malay]{utmthesis}
%options [english, malay, final, draft]

%define whatever packages here
\usepackage{graphicx, cite, url, subfigure, longtable, enumerate, setspace,amsmath, listings, color}
\renewcommand{\lstlistingname}{Senarai}
\special{papersize=210mm,297mm}

%thesis info
\title{Pembangunan Antaramuka Pengguna Bergrafik Untuk Mengautomasikan Konfigurasi Bagi Sistem Terbenam Linux}      %thesis/dissertation/project report/repot title
\author{Muhammad Fauzilkamil Bin Zainuddin}             %The author
\degree{Sarjana Muda Sains Komputer}                    %The official degree title, e.g., Bachelor in Electrical Engineerig
\faculty{Fakulti Sains Komputer Dan Sistem Maklumat}    %The faculty
\titledate{Februari 2011}                                %Month submitted
\award{1}                                               %1. Bachelor Degree Project Report
                                                        %2. Master's Project Report (By course work)
                                                        %3. Master's Dissertation (By course work and research)
                                                        %4. Master's Thesis (By research)
                                                        %5. Doctor of Philosophy Thesis
                                                        %6. Engineering Doctorate Thesis

\superone{Dr. Mohd Shafry Mohd Rahim}       %Your supervisor. Uncomment below if to add more supervisors.
                                            %\supertwo{My Co}
                                            %\superthree{My Another Co}



\begin{document}

%All the prelim pages and such
\preliminary[source/pre]                    %[sample/pre/] is the folder where your tex files are]
                                            %Alternatively, you could manually define which pages to invoke
                                            %\preliminary[pre] is equivalent to:
                                            %\coverpage
                                            %\superpage
                                            %\certification
                                            %\frontmatter
                                            %\maketitle
                                            %\declaration
                                            %\begin{dedication}
\emph{Buat Ayah dan Ibu...
\\Buat Isteri tercinta...
\\Buat anda yang membaca}%
\end{dedication}

                                            %\begin{acknowledgement}

Dengan nama Allah Yang Maha Pemurah lagi Maha Penyayang

Dengan ini, saya ingin menyatakan rasa syukur yang tak terhingga ke hadrat Ilahi atas segala limpah kurnia-Nya maka selesai juga projek akhir ini yang menjadi syarat utama penganugerahan Sarjana Muda Sains Komputer, Fakulti Sains Komputer dan Sistem Maklumat, Universiti Teknologi Malaysia.

Setinggi-tinggi penghargaan kepada Dr. Mohd Shafry Mohd Rahim yang sedia memberi bimbingan dan tunjuk ajar sepanjang perjalanan projek ini. Juga kepada pensyarah-pensyarah yang telah gigih mencurahkan ilmu. Semoga segala ilmu yang diberikan itu, akan dapat saya manfaatkan dengan baik.

Yang sentiasa dalam ingatan, Ayah dan Ibu yang telah banyak memberi sokongan dan galakan, buat Isteri tercinta yang setia memahami dan menyayangi serta buat keluarga yang sentiasa disisi dengan sokongan dan galakan yang berterusan. Terima kasih tak terhingga atas segala jasa dan pengorbanan yang anda semua berikan.

Sekalung penghargaan kepada semua yang terlibat dalam pembikinan dan penghasilan projek ini. Bagi rakan sekerja dan majikan yang memperkenalkan saya dengan elektronik dan sistem automasi kawalan komputer. Buat sifu, guru dan rakan online mahupun offline yang tidak lokek berkongsi ilmu berkaitan Linux, sistem terbenam dan ilmu-ilmu berkaitan. Juga buat rakan-rakan sekelas yang telah banyak membantu. Segala budi dan jasa kalian tetap dikenang selamanya.

\begin{flushright}
\textit{M. Fauzilkamil Zainuddin, Kuala Lumpur}
\end{flushright}
\end{acknowledgement}

                                            %\begin{abstract}
Graphical User Interface or GUI have been used as main interface for computer user for daily usage. Without it, computer user these days will feel odd and restricted to spacific textual commands. However, hardware restrictions in embedded linux system is causing difficulties for this system to have a graphical user interface (GUI). These limitation have made embedded linux system unfriendly to user and can only be used by trained technicians.The use of textual command have been considered primitive and unfriendly to do the settings or configurations of the system. Therefore, Graphical User Interface can be developed to automate the configuration which is normally done using command line interface in embedded linux system. This graphical interface will allow user to do the settings by simple selections using mouse buttons instead of command line which normally have to be remembered and typed. User can also see the status of the system by a simple glance to graphical graphs and tables which will be more friendly. Since the embedded linux system doesn’t have directly connected monitor, this interface will be developed using TCP/IP and HTTP protocols. It will be developed using PHP, Bash, Python, HTML, CSS and JavaScript Programming language.

\end{abstract}

                                            %\begin{abstrak}
Abstract in Malay.Please check whatever relevant terms with DBP.
Follow this URL
\url{http://sbmb.dbp.gov.my/knb/cariankata/dbp_nb_carian_kata_istilah.aspx}

\end{abstrak}

                                            %\tableofcontents%
                                            %\listoftables%
                                            %\listoffigures%
                                            %%List of abbreviation
\listofabbre

%Just add whatever down here.
\begin{acronym}[]
  \acro{AJAX}{Asynchronous Javascript and XML}
  \acro{CLI}{Command Line Interface}
  \acro{HTML}{HyperText Markup Language}
  \acro{SBC}{Single Board Computer}
  \acro{XML}{Extensible Markup Language}
\end{acronym}

%Take note on functionality below. Make full use of it if possible.
%You can insert an abbreviation listed here, e.g. \ac{AJAX} more than once, e.g. \ac{AJAX} into normal text.

                                            %%List of symbols
\listofsymbols

%Just add whatever down here.
\begin{acronym}[]
  \acro{lambda}[$\lambda$]{Wavelength}
  \acro{alpha}[$\alpha$]{Second symbol description}
  \acro{beta}[$\beta$]{Third symbol description}
  \acro{gamma}[$\gamma$]{Fourth symbol description}
\end{acronym}

%Take note on functionality below. Make full use of it if possible.
You can insert a symbol listed here, e.g. \acs{alpha} into normal
text.

                                            %% If your thesis does not have any appendix page,
% comment out the following line to remove the `List of Appendices' page.
\listofappendices



%the main chapters start here.
\mainmatter

%include main chapters. You could use \include here
\input source/main/001-pengenalan.tex
\input source/main/002-kajian-literatur.tex
\input source/main/003-metodologi.tex
\input source/main/004-rekabentuk.tex
\input source/main/005-implementasi.tex
\input source/main/006-pengujian.tex
\input source/main/007-kesimpulan.tex

%these lines define the bibliography
\bibliographystyle{utmthesis-numbering}     %If to use name-year citation, use utmthesis-authordate, else utmthesis-numbering
\bibliography{source/bib/reference}         %where your bib file is
%\nocite{*}

%%Comment all lines below (and % If your thesis does not have any appendix page,
% comment out the following line to remove the `List of Appendices' page.
\listofappendices
 above) when has no appendices
\appendix
%I tried \include but it doesn't work. \input will.
\input source/appendices/lampiran1.tex
\input source/appendices/lampiran2.tex
%The line below is to make sure correct ToC line for appendices.
\endmatter


\end{document}
