\chapter{Pengenalan}\label{c1}%
Pengenalan tesis ini.

\section{Pengenalan}
Sistem terbenam linux merupakan bla bla. merujuk kepada laman wikipedia,

"Sistem terbenam linux adalah blabla"-\cite{w2}.

Whatever you want to mention on context. Not too much, as two or
three paragraphs will do.

\subsection{Pengenalan Kepada Sistem Terbenam Linux}%
Sistem terbenam linux adalah...

\subsection{Pengenalan Kepada PersiaSYS Sdn. Bhd.}%
PersiaSYS Sdn. Bhd. adalah...

\section{Latarbelakang Masalah}
Masalah antaramuka linux sedia ada

\section{Penyataan Masalah}
The general problem in the context that require extended works
presented in this thesis. Better if each problem is explained in a
paragraph. Do citation as~\cite{b1}. You can just cite multiple
citations such as~\cite{j1, c1, w1}. Cite like so \cite{b1}.

\section{Matlamat Projek}
Membangunkan aplikasi antaramuka web

\section{Objektif Projek}
Memudahkan kerja konfigurasi dan kemaskini

\section{Skop Projek}
Terhad kepada:
prototaip

\section{Kepentingan Projek}
Projek ini penting bagi memudahkan kerja dan meningkatkan produktiviti

\section{Struktur Tesis}
Kajian tesis ini dilakukan dan dihuraikan kepada beberapa bab yang menunjukkan perjalanan keseluruhan kajian. Secara ringkasnya bab-bab ini mengandungi:

i. Bab 1 – Pengenalan

Bab ini merupakan proses awal mengenalpasti masalah yang dihadapi dan mengenalpasti peluang dan potensi penyelesaiannya. Lahir dari proses identifikasi ini, halatuju kajian mula dirangka sebagai matlamat, objektif dan skop yang disasarkan untuk dicapai.

ii. Bab 2– Kajian Literasi

Kajian literasi merupakan suatu kajian secara teknikal bagi mengenalpasti komponen-komponen dalam sesebuah pembangunan aplikasi merangkumi perkakasan, perisian, aplikasi setara, justifikasi pemilihan teknologi dan lain-lain yang berkaitan sebagai pemangkin pembangunan.

iii. Bab 3 – Metodologi

Langkah-langkah dan pendekatan yang diambil semasa proses kajian dan pembangunan aplikasi dilakukan diperincikan pada bab ini.

iv. Bab 4 – Rekabentuk

Merupakan bab yang menerangkan proses rekabentuk aplikasi meliputi rekabentuk antaramuka aplikasi, rekabentuk pangkalan data, spesifikasi input/output, rekabentuk logikal dalam bentuk pewakilan model atau rajah, deskripsi dan lain-lain lagi menggunakan pendekatan yang dipilih pada Bab 3.

v. Bab 5 –  Implementasi Aplikasi

Merupakan suatu proses menterjemahkan permasalahan kepada aplikasi boleh guna secara praktikal melibatkan usaha-usaha perlaksanaan seperti pembangunan pangkalan data, modul atau fungsi-fungsi aplikasi, algoritma aplikasi dan sebagainya.

vi. Bab 6 – Pengujian dan Penilaian Aplikasi

Proses pengujian aplikasi terhadap kebolehlarian fungsi-fungsi dan penilaian terhadap aplikasi yang dibangunkan berdasarkan ujian yang dijalankan.

vii. Bab 7 – Perbincangan dan Kesimpulan

Bab ini merupakan bab yang membincangkan isu-isu khusus aplikasi sama ada potensi aplikasi ini sebagai alternatif penyelesaian kepada permasalahan sedia ada, kelemahan-kelemahan yang wujud untuk penambahbaikan dan kesimpulan yang dapat dirumuskan berdasarkan kajian yang dilakukan.