\chapter{Kajian Literatur}\label{c1}%
Pendahuluan kajian literatur.

\section{Pengenalan}
Pengenalan Kajian Literatur

\section{Kajian Kepada Antaramuka Sistem Terbenam Linux}
Antaramuka sistem terbenam linux sedia ada

\subsection{Kajian Sistem atau Aplikasi Semasa}

\subsection{Kajian Terhadap Penyelesaian Semasa}

\section{Kajian Teknologi}

\subsection{Teknologi Perisian}

\subsubsection{Sistem Terbenam Linux}
debian-arm
redboot

\subsubsection{Pelayan Web Apache2}
Apache ialah apa...

\subsubsection{PHP}
PHP ialah apa...

\subsubsection{Bash}
Bash ialah apa...

\subsubsection{Python}
Python ialah apa...

\subsubsection{HTML}
HTML ialah apa...

\subsubsection{JavaScript}
JavaScript ialah apa...

\subsubsection{CSS}
CSS ialah apa...

\subsubsection{SVG}
SVG ialah apa...

\subsection{Teknologi Perkakasan Komputer}
Perkakasan yang digunakan dalam projek ini

\subsubsection{\textit{Single Board Computer}}
Komputer papan tunggal

\subsubsection{Komputer Peribadi atau Laptop}
boleh guna laptop atau PC

\subsubsection{Sambungan Rangkaian bagi Sistem Tanpa Kepala}
dengan sambungan rangkaian

\section{Kesimpulan}
kesesuaian, inovasi, kelebihan, kemudahan, kos
