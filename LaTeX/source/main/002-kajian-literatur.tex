\chapter{Kajian Literatur}\label{c1}%

\section{Pengenalan}
Kajian literatur merupakan suatu kajian terhadap keperluan pembangunan projek. Semasa kajian ini dilakukan, segala maklumat-maklumat berkaitan telah dikumpulkan bagi memahami dan mengenalpasti masalah-masalah yang terlibat, kekangan-kekangan dalam pembangunan, teknologi perkakasan dan perisian yang sesuai, dan lain-lain lagi. Kaedah kajian literatur perlu dilakukan dengan teliti kerana ianya adalah amat penting untuk merangka
perancangan penyelesaian bijak bagi menjamin kelancaran pelaksanaan kajian dan seterusnya menjamin keberkesanan aplikasi yang bakal dibangunkan kerana kegagalan mengenalpasti masalah-masalah yang berkaitan semasa peringkat kajian literatur ini boleh membawa kepada kegagalan kajian secara keseluruhannya.

\section{Kajian Kepada Antaramuka Sistem Terbenam Linux}
Antaramuka sedia ada dalam sistem terbenam Linux Realtime Rain Gauge terhad kepada penggunaan Antaramuka Baris Perintah atau lebih dikenali sebagai CLI sebagai singkatan kepada \textit{Command Line Interface}. Ini bagi meminimakan penggunaan sumber di dalam sistem yang padat dan kecil. Antaramuka CLI telah digunakan bagi melakukan konfigurasi dan penyeliaan sistem terbenam ini hingga sekarang.

Antaramuka Baris Perintah adalah mekanisma interaksi dengan sistem operasi dengan menaip arahan untuk menjalankan sesuatu tugas.

\subsection{Kajian Sistem atau Aplikasi Semasa}

\subsection{Kajian Terhadap Penyelesaian Semasa}

\section{Kajian Teknologi}

\subsection{Teknologi Perisian}

\subsubsection{Sistem Terbenam Linux}
debian-arm
redboot

\subsubsection{Pelayan Web Apache2}
Apache ialah apa...

\subsubsection{PHP}
PHP ialah apa...

\subsubsection{Bash}
Bash ialah apa...

\subsubsection{Python}
Python ialah apa...

\subsubsection{HTML}
HTML ialah apa...

\subsubsection{JavaScript}
JavaScript ialah apa...

\subsubsection{CSS}
CSS ialah apa...

\subsubsection{SVG}
SVG ialah apa...

\subsection{Teknologi Perkakasan Komputer}
Perkakasan yang digunakan dalam projek ini

\subsubsection{\textit{Single Board Computer}}
Komputer papan tunggal

\subsubsection{Komputer Peribadi atau Laptop}
boleh guna laptop atau PC

\subsubsection{Sambungan Rangkaian bagi Sistem Tanpa Kepala}
dengan sambungan rangkaian

\section{Kesimpulan}
kesesuaian, inovasi, kelebihan, kemudahan, kos
