\chapter{Kesimpulan}\label{c7}
\section{Pengenalan}
Antaramuka Pengguna Bergrafik Untuk Mengautomasikan Konfigurasi Bagi Sistem Terbenam Linux ini dibina dengan tujuan mempermudah dan mempercepatkan tugas konfigurasi dan penyelenggaraan sistem terbenam. Dengan adanya antaramuka ini, ia membolehkan juruteknik biasa yang tidak memiliki kepakaran Linux untuk melakukan tugas rutin penyelenggaran dengan mudah seterusnya dapat meningkatakan produktiviti, mengurangkan kos penyelenggaraan dan menjimatkan masa.

\section{Hasil dan Pencapaian}
Kajian menunjukkan bahawa peningkatan kadar produktiviti dan penjimatan masa yang memberansangkan dapat dicapai dalam tugas penyelenggaraan sistem yang banyak. Juruteknik biasa boleh dihantar bagi melakukan kerja-kerja penyelenggaraan dan mereka dapat melakukannya dengan lebih mudah dan cepat. Kesilapan menaip arahan dapat dikurangkan ke tahap paling minima dengan susunan menu dan penerangan jelas mengenai sesuatu tetapan yang boleh dilkukan dalam antaramuka bergrafik ini.

\section{Kelebihan Aplikasi}


\section{Kelemahan Aplikasi}


\section{Cadangan Penambahbaikan}
Selain daripada boleh digunakan untuk membuat tetapan atau konfigurasi untuk beberapa bahagian penting dalam sistem ini, kaedah antaramuka bergrafik ini boleh dikembangkan untuk fungsi yang generik atau spesifik terhadap produk sistem terbenam yang disasarkan oleh pembangun. Selain dari itu, sistem antaramuka dengan kaedah ini boleh ditingkatkan lagi dan dikompil sebagai:
- Sebuah kerangka (framework) untuk aplikasi web bagi sistem terbenam Linux yang lain seperti pakej-pakej sumber seumpama Wordpress untuk blog, Joomla untuk portal serta laman web dan juga projek yang melibatkan sistem terbenam linux.
- Fungsi-fungsi baru boleh dihasilkan dan ditambah kedalam sistem ini dalam bentuk modul supaya ianya dapat digunakan apabila perlu dan dibatalkan apabila tidak diperlukan.

\section{Kesimpulan}
Daripada kajian ini, dapatlah disimpulkan bahawa Antaramuka Pengguna Bergrafik telah dapat mempermudah dan mempercepatkan tugas konfigurasi sistem terbenam Linux. 
