\chapter{Implementasi}\label{c5}

\section{Pengenalan}
Fasa implementasi merupakan suatu proses perlaksanaan pembangunan aplikasi hasil dari proses rekabentuk yang telah dilakukan. Proses ini melibatkan perlaksanaan pembangunan pangkalan data, algoritma aturcara dan pembangunan modul atau fungsi aplikasi secara praktikal untuk menyelesaikan masalah.

Secara ringkasnya, proses implementasi melibatkan proses-proses dan jujukan-jujukan algoritma aplikasi yang berkomunikasi untuk tujuan pengoperasian yang cekap secara kolektif sebelum data yang hendak dimasukkan/dikemaskini dimuatkan ke dalam pangkalan data dan hasil akhir yang terkemaskini pada pangkalan data seharusnya memenuhi tuntutan dan matlamat penyelesaian yang hendak dicapai.

\section{Pemasangan Pelayan Web dan PHP}
Sebelum sesebuah sistem terbenam dapat menyediakan layanan web, ia mestilah dipasang dengan pelayan web yang menyokong PHP. Berikut adalah kaedah untuk memasang pelayang web pada sistem terbenam Linux Rain Gauge yang berasaskan Debian ARM ini.

{\footnotesize\renewcommand{\baselinestretch}{1.0}
\begin{verbatim}
   root@raingauge:~# apt-get install lighttpd
\end{verbatim}}

\section{Pembangunan Antaramuka}

\section{Skrip Menghubungkan Antaramuka Web Dengan Arahan Linux}

\section{Pengaturcaraan Fungsi Aplikasi}

\subsection{Fungsi Set IP}
Fungsi set IP ditulis...

\begin{figure*}[!ht]
\lstset{ %
caption=Fungsi Set IP menggunakan PHP, %the caption
label=l2,                       % the label
language=PHP,                % choose the language of the code
basicstyle=\footnotesize,       % the size of the fonts that are used for the code
numbers=left,                   % where to put the line-numbers
numberstyle=\footnotesize,      % the size of the fonts that are used for the line-numbers
stepnumber=2,                   % the step between two line-numbers. If it's 1 each line will be numbered
numbersep=5pt,                  % how far the line-numbers are from the code
%backgroundcolor=\color{cyan},   % choose the background color. You must add \usepackage{color}
showspaces=false,               % show spaces adding particular underscores
showstringspaces=false,         % underline spaces within strings
showtabs=false,                 % show tabs within strings adding particular underscores
frame=single,                   % adds a frame around the code
tabsize=2,                      % sets default tabsize to 2 spaces
captionpos=b,                   % sets the caption-position to bottom
breaklines=true,                % sets automatic line breaking
breakatwhitespace=false %,        % sets if automatic breaks should only happen at whitespace
%escapeinside={\*}{*)}          % if you want to add a comment within your code
}

\lstinputlisting{source/appendices/set_ip.php}
\end{figure*}

\subsection{Fungsi Set GPRS Provider}

\subsection{Fungsi Set Device ID}

\subsection{Fungsi Melihat Log Status}

\subsection{Fungsi Melihat Realtime Data}

\subsection{Fungsi-fungsi Tambahan}

\section{Pengintegrasian Fungsi Aplikasi}

\section{Kesimpulan}


