\begin{abstrak}
Antaramuka Pengguna Bergrafik atau GUI telah digunakan sebagai antaramuka utama bagi pengguna komputer untuk tugas-tugas harian. Tanpanya, pengguna hari ini akan berasa kekok kerana terhad kepada arahan teks yang tertentu. Namun bagi sistem terbenam linux, antaramuka pengguna bergrafik ini tidak wujud sebagai antaramuka utama kerana kaedah paparan dan akses terhad yang biasanya sistem ini tidak dilengkapi sambungan monitor dan tetikus untuk tujuan tersebut. Kekangan ini menjadikan sistem terbenam linux tidak mesra pengguna dan hanya boleh digunakan oleh juruteknik yang terlatih sahaja. Meskipun begitu, perkembangan teknologi yang pesat pada hari ini telah melerai kekangan-kekangan pengkomputeran masa lalu. Kewujudan teknik dan teknologi perkakasan dan perisian terkini telah membantu menyelesaikan permasalahan komputer sebelum ini. Tidak ketinggalan juga, sistem terbenam linux turut mengalami evolusi ini. Arahan-arahan teks untuk konfigurasi sistem terbenam linux boleh diautomasikan dan dilarikan melalui antaramuka pengguna bergrafik dengan menggunakan perkakasan dan perisian terkini. Untuk memudahkan penggunaan sistem terbenam linux, saya telah memilih untuk membangunkan antaramuka pengguna bergrafik untuk mengautomasikan konfigurasi bagi sistem terbenam linux. Dalam projek ini, saya akan menunjukkan bagaimana kita dapat membangunkan antaramuka pengguna bergrafik untuk mengautomasikan konfigurasi bagi sistem terbenam linux dengan menggunakan idea, teknik, dan teknologi baru.
\end{abstrak}
