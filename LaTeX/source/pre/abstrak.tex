\begin{abstrak} Antaramuka Pengguna Bergrafik atau GUI telah digunakan sebagai antaramuka utama bagi pengguna komputer untuk tugas-tugas harian. Tanpanya, pengguna hari ini akan berasa kekok kerana terhad kepada arahan teks yang tertentu. Namun, kekangan peranti yang terhad bagi sistem terbenam linux menimbulkan kesukaran untuk mewujudkan antaramuka pengguna bergrafik (GUI) untuknya. Kekangan ini menjadikan sistem terbenam linux tidak mesra pengguna dan hanya boleh digunakan oleh juruteknik yang terlatih sahaja. Penggunaan arahan teks sudah dianggap ketinggalan dan tidak mesra pengguna bagi membuat tetapan atau konfigurasi sistem. Justeru itu, Antaramuka Pengguna Bergrafik boleh dibangunkan sebagai antaramuka untuk mengautomasikan konfigurasi yang biasanya dilakukan menggunakan antaramuka baris arahan bagi sistem terbenam linux. Antaramuka bergrafik ini membolehkan pengguna membuat tetapan dengan membuat pilihan mudah menggunakan butang tetikus berbanding baris arahan yang perlu dihafal dan ditaip. Pengguna juga dapat melihat status sistem dengan sekali imbas pada graf dan jadual bergrafik yang cantik dan lebih mesra. Oleh kerana sistem terbenam linux tidak mempunyai sambungan terus kepada monitor, antaramuka ini akan dibangunkan menggunakan protokol TCP/IP dan HTTP. Ia akan dibangunkan menggunakan bahasa pengaturcaraan PHP, Bash, Python, HTML, CSS dan Javascript.

\end{abstrak}
